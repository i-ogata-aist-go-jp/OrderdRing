\documentclass{article}
\bibliographystyle{plain} 

\usepackage[utf8]{inputenc} % set input encoding (not needed with XeLaTeX)

%%% Examples of Article customizations
% These packages are optional, depending whether you want the features they provide.
% See the LaTeX Companion or other references for full information.

%%% PAGE DIMENSIONS
\usepackage{geometry} % to change the page dimensions
\geometry{a4paper} % or letterpaper (US) or a5paper or....
% \geometry{margins=2in} % for example, change the margins to 2 inches all round
% \geometry{landscape} % set up the page for landscape
%   read geometry.pdf for detailed page layout information

\usepackage{graphicx} % support the \includegraphics command and options

% \usepackage[parfill]{parskip} % Activate to begin paragraphs with an empty line rather than an indent

%%% PACKAGES
\usepackage{booktabs} % for much better looking tables
\usepackage{array} % for better arrays (eg matrices) in maths
\usepackage{paralist} % very flexible & customisable lists (eg. enumerate/itemize, etc.)
\usepackage{verbatim} % adds environment for commenting out blocks of text & for better verbatim
\usepackage{subfig} % make it possible to include more than one captioned figure/table in a single float
% These packages are all incorporated in the memoir class to one degree or another...

%%% SECTION TITLE APPEARANCE
\usepackage{sectsty}
\allsectionsfont{\sffamily\mdseries\upshape} % (See the fntguide.pdf for font help)
% (This matches ConTeXt defaults)

%%% ToC (table of contents) APPEARANCE
\usepackage[nottoc,notlof,notlot]{tocbibind} % Put the bibliography in the ToC
\usepackage[titles,subfigure]{tocloft} % Alter the style of the Table of Contents
\renewcommand{\cftsecfont}{\rmfamily\mdseries\upshape}
\renewcommand{\cftsecpagefont}{\rmfamily\mdseries\upshape} % No bold!


\usepackage{amsmath,amssymb,amsthm}
\usepackage{cmll,braket,stmaryrd}
\usepackage{backnaur}
\usepackage{url}

\usepackage{forest}
\usepackage{tikz-qtree}
\usetikzlibrary{matrix}
\usepackage{bussproofs}  


%%% END Article customizations

\theoremstyle{definition}
\newtheorem{theorem}{Theorem}[section]
\newtheorem{lemma}[theorem]{Lemma}
\newtheorem{proposition}[theorem]{Proposition}
\newtheorem{fact}[theorem]{fact}
\newtheorem{corollary}[theorem]{Corollary}
\newtheorem{definition}[theorem]{Definition}
\newtheorem{example}[theorem]{Example}
\newtheorem{remark}[theorem]{Remark}

%% newcommands

\newcommand{\defeq}{\mathrel{\mathop:}=}
\newcommand{\eqdef}{\mathrel{\mathop=}:}
\newcommand{\QTargument}{{(\mbox{\textendash})}}
\newcommand{\ONE}{\mbox{\boldmath{1}}}
\newcommand{\ZERO}{\mbox{\boldmath{0}}}

\begin{document}

\section{Partial and Total Order}

\begin{definition} [Cartesian Product]
The Cartesian product of two sets $A$ and $B$ 
is defined to be the set of all points $(a,b)$
where $a \in A$ and $b \in B$. 
It is denoted $A \times B$.
\end{definition}

\begin{definition} [Binary Relation]
Let $S$ be a set of objects.
A binary relation on $S$ is a subset of the Cartesian product $S \times S$.
\end{definition}

\begin{definition}[preorder]
Let $A$ be a set. A {\em preorder} on $A$ is a binary relation $\leq$ which is
\begin{enumerate}
\item {\em reflexive}: for all $a \in A$,  $a \leq a$,
\item {\em transitive}:  if $a \leq b$ and $b \leq c$, then $a \leq c$, and
\end{enumerate}
\end{definition}

A partial order is a preorder which is antisymmetric.

\begin{definition}[partial order]
Let $A$ be a set. A {\em partial order} on $A$ is a binary relation $\leq$ which is
\begin{enumerate}
\item {\em reflexive}: for all $a \in A$,  $a \leq a$,
\item {\em transitive}:  if $a \leq b$ and $b \leq c$, then $a \leq c$, and
\item {\em antisymmetric}: if $a \leq b$ and $b \leq a$, then $a = b$.
\end{enumerate}
\end{definition}

\begin{definition}[total preorder]
A {\em total preorder} on $A$ is a binary relation $\leq$ such that; 
\begin{enumerate}
\item {\em transitive}:  if $a \leq b$ and $b \leq c$, then $a \leq c$, and
\item {\em total}: for all $a,b \in A$, 
one of $a \leq b$,$b \leq a$ holds. 
\end{enumerate}
From totality, we have reflexivity ($ \forall a \in A, a \leq a$). 
\end{definition}

A total order is a total preorder which is antisymmetric.
In other words, a total order is a total preorder which is also a partial order.

\begin{definition}[total order]
A {\em total order} on $A$ is a binary relation $\leq$ such that; 
\begin{enumerate}
\item {\em partial order}: $(A,\leq)$ is a poset,
\item {\em total}: for all $a,b \in A$, 
one of $a \leq b$,$b \leq a$ holds. 
\end{enumerate}
Then a double $(A,\leq)$ is called a totally ordered set. 
\end{definition}

From the point of view of proof,
the reflexive rule is axiom, while the transitive and the antisymmetrc rules are inference rules. 

In this article, these axioms and inference rules are represented pictorially as; 

\begin{center}
\begin{prooftree}
	\AxiomC{} \UnaryInfC{$a \leq a$}
\DisplayProof  \hskip 2cm
	\AxiomC{$a \leq b$} \AxiomC{$b \leq c$}
    \BinaryInfC{$a \leq c$}
\DisplayProof  \hskip 2cm
	\AxiomC{$a \leq b$} \AxiomC{$b \leq a$}
    \BinaryInfC{$a = b$}
\end{prooftree}
\end{center}

\begin{definition} [poset and its opposite]
Let $A$ be a set and $\leq$ be a partial order. 
A {\em poset} (short for partially ordered set) is defined to be a double  $(A,\leq)$.
%
Its {\em opposite}, denoted $(A^{\sf op},\geq)$, 
is the poset with the same underlying set, with $a \geq b$ in $A^{\sf op}$ 
iff $a \leq b$ in $A$. 
\end{definition}

\subsection{Commutative Monoids and Groups}
% magma
\begin{definition} [commutative magma]
A {\em commutative magma} is a basic kind of (commutative) algebraic structure. 
It consists of a set $M$ accompanied with a binary operation "$\cdot$" 
which is a map: 
$ {\QTargument} { \cdot } {\QTargument} : M \cdot M \longrightarrow M$.  
That is, it sends any two elements $a,b \in M$ to another element $a \cdot b$.
%
In set theoretic notation:
\begin{enumerate}
\item {\em closure law}:   $ \forall a,b \in M, a \cdot b \in M $.
\item {\em commutative law}:  for every $a,b \in M$ we have $a \cdot b = b \cdot a$.
\end{enumerate}
\end{definition}

\begin{definition}[commutative semigroup]
A {\em commutative semigroup} is a triple 
$(S,\cdot)$. 
The following conditions hold:
%
\begin{enumerate} 
\item {\em commutative magma}:  $(S,\cdot)$ is a commutative magma,
\item {\em associative law}:  for every $a,b,c \in S$,
we have $a \cdot ( b \cdot c) =  (a \cdot b) \cdot c$.
\end{enumerate}
\end{definition} 

\begin{definition}[commutative monoid]
A {\em commutative  monoid} is a triple 
$(M,\cdot,\ONE)$  where $\ONE \in M$. 
The following conditions hold:
%
\begin{enumerate} 
\item {\em commutative semigroup}:  $(M,\cdot)$ is a commutative semigroup,
\item {\em neutrality law}:  for all  $a \in M$  we have $\ONE \cdot a = a$.
\end{enumerate}
\end{definition} 

\begin{definition}[commutative group]
A {\em commutative group} is a triple 
$(G,\cdot,\ONE) $  where $\ONE \in G$. 
The following conditions hold:
%
\begin{enumerate} 
\item {\em commutative monoid}:  $(G,\cdot,\ONE) $ is a commutative monoid.
\item {\em inverse element}:  for each $a \in G$,
we have an inverse element $b \in G$ 
such that $a \cdot b =  b \cdot a = \ONE$.
\end{enumerate}
\end{definition} 

\section{Ordered Ring}

\begin{definition}[commutative semiring]
A commutative (i.e. $a \otimes b = b \otimes a$)
{\em semiring} is a quintuple $(R,+,0,\otimes,1) $.
The following conditions hold:
%
\begin{enumerate} 
\item {\em additive commutative monoid}:
$(R,+,0)$ is a commutative monoid with identity element $0$,
\item {\em multiplicative commutative monoid}:
$(R,\otimes,1)$ is a commutative monoid with identity element $1$, 
\item {\em anihilative law}:  $a \otimes 0 = 0$,
\item {\em distributive law}:
$\otimes$ distributes over $+$;
\[ c \otimes (a + b) = (c+a) \otimes (c+b) \]
\end{enumerate}
\end{definition}


The difference between rings and semirings
is that addition yields only a commutative monoid, 
not necessarily a commutative group. Explicitly,

\begin{definition}[commutative ring]
A {\em commutative ring} is a quintuple $(R,+,0,\otimes,1) $.
The following conditions hold:
%
\begin{enumerate} 
\item {\em abelian group}:
$(R,+,0)$ is an abelian group(commutative group) with identity element $0$,
\item {\em multiplicative commutative monoid}:
$(R,\otimes,1)$ is a commutative monoid with identity element $1$, 
\item {\em distributive law}:
$\otimes$ distributes over $+$;
\[ c \otimes (a + b) = (c+a) \otimes (c+b) \]
\end{enumerate}
\end{definition}

\begin{fact}
$a \otimes 0 = 0$ (anihilative law) follows from the other ring conditions.
\end{fact}
\begin{proof}
First, we show that $b = b+b$ (idempotency) implies $b = 0$
in groups. 
If $c$ is an inverse of $b$ 
then $ 0 = b + c = (b + b) + c = b + (b + c) = b + 0 = b$.
Thus 
$ a \otimes 0 = a \otimes (0 + 0) =  (a \otimes 0) + (a \otimes 0) $
implies $a \otimes 0 = 0$. 
\end{proof}

\begin{fact}
If $c$ is an inverse of $b$ 
then $a \otimes c$ is an inverse of $a \otimes b$. 
\end{fact}
\begin{proof}
\[ (a \otimes b) + (a \otimes c) = a \otimes (b + c) = a \otimes 0 = 0 \].
\end{proof}

\begin{definition}[field]
A {\em field} is a quintuple $(R,+,0,\otimes,1)$.
The following conditions hold:
%
\begin{enumerate} 
\item {\em commutative ring}:
$(R,+,0,\otimes,1)$ is a commutative ring,
\item {\em multiplicative inverse}: every nonzero element 
has a multiplicative inverse. 
\end{enumerate}
\end{definition}


\subsection{Ordered Commutative Groups and Rings}


\begin{definition} [Ordered Abelian Groups]
A {\em ordered abelian groups} is a commutative group $G$
with a total order $\leq$  such that for all $a,b,c \in G$,
\begin{enumerate}
\item {\em order preserving}: if $a \leq b$ then $a + c \leq b + c$.
\end{enumerate}
\end{definition}

Every Archimedean group (a bi-ordered group satisfying an Archimedean property) 
is isomorphic to a subgroup of the additive group of real numbers (\cite{fuchs2001modules}, p.61 ).
The proof appears in  \cite{Fuchs1983}.

\begin{definition} [Ordered rings]
A {\em commutative ordered ring} is a commutative ring $R$
with a total order $\leq$  such that for all $a,b,c \in R$,
\begin{enumerate}
\item {\em order preserving}: if $a \leq b$ then $a + c \leq b + c$.
\item {\em positiveness} if $0 \leq a$ and $0 \leq b$ then $0 \leq a \otimes b$.
\end{enumerate}
\end{definition}
%
\bibliography{BiB/algebra}
%
\end{document}


